\section{Conclusions an Future Work}
  \label{sec:conc}

  We presented the first study of using multiple queries to represent
  an information need in the cluster-based document retrieval realm.

  We first studied the cluster hypothesis using multiple queries. We
  found that the cluster hypothesis holds to a larger extent,
  according to a specific test, for highly effective queries than for
  less effective queries. This attests to the fact that top-retrieved
  relevant documents are less similar to each other with increasing effectiveness of the query used for retrieval.

  We then studied the {\em optimal cluster} phenomenon with respect to
  the effectiveness of the query used for retrieval. The optimal
  cluster is a cluster of similar documents which contains the highest
  percentage of relevant documents with respect to clusters of similar
  documents induced from top-retrieved documents. We found that the percentage of relevant documents in the optimal cluster increases with increased effectiveness of the query used for retrieval.

  Another important finding of our analysis was that the relative
  effectiveness of cluster-based document retrieval methods can change
  with respect to the effectiveness of the (single) query used for
  retrieval. Furthermore, we showed that state-of-the-art
  cluster-based retrieval methods can, in fact, underperform standard document retrieval if highly effective queries are used.

  We proposed a suite of retrieval templates that allow to adapt
  existing cluster-based document retrieval methods which were
  designed to work with a single query to utilize multiple
  queries. The templates can be instantiated to yield specific
  retrieval methods by selecting, for example, the cluster-based
  retrieval method to use. The templates leverage information induced from the multiple queries at the query level, document list level (by fusion), and feature level --- namely, document-query and cluster-query similarity estimates.

  Extensive empirical evaluation showed that the best performing
  methods instantiated from the templates are highly effective,
  specifically, with respect to very strong baselines of using multiple queries and state-of-the-art cluster-based document retrieval methods that utilize a single query.

  For future work we plan to explore the development of additional
  templates and study the instantiations of templates with additional
  clustering algorithms, cluster-based retrieval methods and fusion
  techniques.
