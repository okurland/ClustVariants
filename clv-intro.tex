\section{introduction}
\label{sec:intro}

The cluster hypothesis for ad hoc retrieval is: ``{\em closely
associated documents tend to be relevant to the same
requests}''~{\cite{Jardine+Rijsbergen:71a}}.
The operational interpretation of the hypothesis is that relevant
documents are more similar to each other than to non-relevant
documents.
This foundational principle of IR has been studied extensively over
the years in order to better understand the implications and
applications of the
hypothesis~{\cite{Jardine+Rijsbergen:71a,Voorhees:85a,elHamdouchi+Willett:87a,Crestani+Wu:06a,Smucker+Allan:09a,Raiber+Kurland:14c}}.
Work in this area has resulted in several inter-document similarity
measures were devised to improve the cluster hypothesis test
results~\cite{Tombros+Villa+Rijsbergen:02a,Raiber+al:15a}, and even a
``reversed'' cluster hypothesis which states that documents
co-relevant to queries should also be similar~\cite{Fuhr+al:12a}.

On the operational side, the cluster hypothesis has driven a huge
body of work on document retrieval methods that leverage information
induced from clusters of similar documents.
There are two main categories of cluster-based document retrieval
methods~\cite{Kurland+Lee:04a}.
The first includes methods that rank document clusters by the
estimated percentage of relevant documents they contain; then, the
cluster ranking is transformed to document
ranking~\cite{Jardine+Rijsbergen:71a,Croft:80a,Voorhees:85a,elHamdouchi+Willett:89a,Tombros+Villa+Rijsbergen:02a,Kurland+Lee:04a,Liu+Croft:04a,Kurland+Lee:06a,Liu+Croft:06a,Liu+Croft:06b,Liu+Croft:08a,Kurland+Domshlak:08a,Kurland+Krikon:11a,Raiber+Kurland:13a}.
A document retrieval method based on supervised ranking of document
clusters was the best performing in the Web track of TREC
2013~\cite{Thompson+al:13a}.

One of the main motivations for pursuing the cluster ranking task is
the {\em optimal cluster}
phenomenon~\cite{Cutting+al:92a,Tombros+Villa+Rijsbergen:02a,Liu+Croft:06b,Kurland+Domshlak:08a}:
if the documents most highly ranked by an initial search are
clustered, these clusters tend to contain a very high percentage of
relevant documents, and the cluster with the highest percentage is
denoted as ``the optimal cluster''.
Positioning the constituent documents of the optimal cluster at the
top of the final result list results in higher overall precision
scores than other commonly used document-based retrieval
methods~\cite{Tombros+Villa+Rijsbergen:02a,Liu+Croft:06b,Kurland+Domshlak:08a}.
A case in point, for a standard TREC corpus, the average precision at
top five (P@5) that results from identifying the optimal
cluster can reach ~$0.8$ in comparison to $~0.46$ attained by a
standard language model retrieval
\cite{Kurland+Domshlak:08a}\footnote{These performance numbers were
reported for TREC's AP corpus with the titles of topics 51-150
serving as queries \cite{Kurland+Domshlak:08a}.}.

The second category of cluster-based document retrieval methods
includes those that enrich the document representation with
information induced from the clusters
\cite{Singhal+Pereira:99a,Kurland+Lee:04a,Liu+Croft:04a,Kurland:09a};
e.g., in the language modeling framework, document language models
can be smoothed using cluster language models
\cite{Kurland+Lee:04a,Liu+Croft:04a,Kurland:09a}.
The motivation is to reduce potential vocabulary mismatches between
queries and relevant documents by enriching a document
representation with information induced from similar documents.

As in the vast majority of work on ad hoc document retrieval, work on
the cluster hypothesis and cluster-based document retrieval was
confined to the standard setting of a single query representing an
information need.
However, it has been recognized for some time that combining multiple
queries which represent the same information need can dramatically
improve retrieval effectiveness~\cite{Belkin+al:93a,Belkin+al:95a}.
Recently, there has been a renewed interest in the importance of
multiple queries retrieval
setting~\cite{bailey2016uqv100,bailey2017retrieval,bc17-adcs,Thomas+al:17a,Benham+al:19a,Liu+al:19a,Lu+al:19a,Zendel+al:19a},
with growing evidence to its operational feasibility and importance.

The feasibility of operationalizing this idea was initially explored
nearly a decade ago in the Bing search engine~{\cite{Sheldon+al:11a}}.
It was recently shown that query variations automatically selected
from a query log of a commercial search engine can be, on average,
effective to the same extent as human generated query
variations~\cite{Liu+al:19a}.
Other recent work on using multiple queries for retrieval has focused
primarily on fusion of the retrieved document
lists~\cite{Belkin+al:93a,Belkin+al:95a,Pickens+al:08a,bailey2017retrieval,bc17-adcs}, relevance modeling (query expansion)~\cite{Lu+al:19a}, and
query performance prediction~{\cite{Zendel+al:19a}}.
Last year, Microsoft released a public API of highly effective
contextual query embeddings which can be used for a variety of
retrieval tasks which are constructed automatically clustering
similar queries in Bing search logs~{\cite{zhang2019generic}} using
a random walk over click
graphs~{\cite{craswell2007random,Sheldon+al:11a,Liu+al:19a}}.

In this paper, we explore the multiple-queries setting in the
cluster-based document retrieval realm.
To the best of our knowledge, there are no previous studies of
cluster-based document retrieval or the cluster hypothesis with
multiple queries.
We propose a suite of {\em retrieval templates} that adapt existing
cluster-based retrieval approaches, which were originally designed to
work with a single query, to use multiple queries representing the
same information need.
We use the terminology ``template'' as each model can be instantiated
in various ways to yield specific retrieval methods; e.g., by
selecting the clustering algorithm and the specific cluster-based
retrieval method to adapt.
The framework can be applied to a large number of cluster-based
retrieval methods, including methods that are based on cluster
ranking and methods that enrich document representations using
cluster-based information.
Our adaptation techniques operate at the query level, document-list
level and feature level.

Extensive empirical evaluation validate the effectiveness gains
achievable when using retrieval templates for cluster-based
retrieval.
The resultant methods substantially outperform state-of-the-art
cluster-based retrieval using a single query as is standard.
The performance of our methods also substantially exceeds that of the
strongest baselines reported in the literature for utilizing multiple
queries for retrieval.

In addition, we test the cluster hypothesis with queries of varying
effectiveness and arrive to an additional interesting finding: for
poorly performing queries the cluster hypothesis holds to a larger
extent than for highly effective queries.
That is, using highly effective queries for retrieval yields result
lists where relevant documents are less similar to each other than
in the result lists retrieved for poorly performing queries.
\shane{This may actually be a result of variant fusion which improves
diversification as in the de Rijke paper.}
This finding helps to explain another important, and somewhat
striking finding that has emerged in our analysis: state-of-the-art
cluster-based retrieval methods can be less effective than standard
document retrieval when using highly effective queries.
However, the merits of cluster-based document retrieval still hold in
the vast majority of cases where the initial query provided to the
retrieval system is of average to poor effectiveness.

An additional important finding in our analysis is that the percentage
of relevant documents in an optimal cluster is dramatically affected
by query effectiveness.
A case in point, the optimal cluster numbers reported in the past for
TREC's topic titles \cite{Liu+Croft:06b,Kurland+Domshlak:08a} can be
dramatically different for different queries representing the same
information needs (topics).

Our contributions can be summarized as follows:
\begin{itemize}
\item This is the first study of the cluster-based document retrieval
realm in a retrieval setting with multiple queries representing the
same information need.
\item We study the cluster hypothesis with queries of varying
effectiveness and empirically show that for highly effective queries
the test holds to a lower degree than for moderately and poorly
performing queries.
\item We empirically demonstrate that the relative effectiveness of
state-of-the-art cluster-based document retrieval methods with
respect to standard document-based retrieval methods can be
dramatically affected by the effectiveness of the query used.
\item We propose a suite of retrieval templates that allow to adapt
various cluster-based retrieval methods to utilize multiple queries.
The resultant effectiveness of retrieval methods instantiated from
these templates transcends the state-of-the-art in (i) cluster-based
retrieval using a single query and (ii) prior work on utilizing
multiple queries for retrieval.
\end{itemize}

