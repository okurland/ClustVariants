\section{Related Work}
\label{sec:rel}
The merits of applying fusion to document lists retrieved for queries
representing the same information need have long been demonstrated
\cite{Belkin+al:93a,Belkin+al:95a,Pickens+al:08a,bailey2017retrieval,bc17-adcs,Benham+al:19a}. One
of the retrieval templates we propose uses cluster-based retrieval to
re-rank a list fused from those retrieved for the queries at hand. We
also present the merits of applying fusion after each of the lists
retrieved for the queries was re-ranked using cluster-based
retrieval. Benham and Culpepper \cite{bc17-adcs} showed that
reciprocal rank fusion \cite{Cormack+al:09a} yields state-of-the-art
performance in fusing document lists retrieved for multiple queries representing the same information need . Our best
performing methods substantially outperform this fusion approach.

Recent work on query performance prediction showed that the relative
prediction quality posted by prediction methods can significantly vary
when varying the effectiveness of a query used to represent an
information need \cite{Zendel+al:19a}. In a conceptually similar vein, we show that state-of-the-art
cluster-based document retrieval methods can actually be underperformed by
simple standard document retrieval if highly effective queries are used to
represent information needs.

To address search efficiency when using multiple queries, Benham et
al. \cite{Benham+al:19a} clustered queries offline. At query time, the
results retrieved for a (single) query at hand were fused with those
for the centroid of the cluster with which the query was
associated. Our work focuses on using document clusters
at query time in contrast to utilizing (offline) query clusters.

There was some work on cluster-based fusion of retrieved lists
\cite{Kozorovitzky+Kurland:11b}. Documents in lists retrieved by applying different
retrieval approaches for a {\em single query} were
clustered. Then, a cluster-based retrieval approach was applied using
these clusters to perform the fusion. This fusion approach is easily derived as a specific instantiation of one of the retrieval templates we present for retrieval using {\em multiple queries}. While the resultant performance is effective, we present approaches which are substantially more effective in our setting.







