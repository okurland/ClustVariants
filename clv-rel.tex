\section{Related Work}
\label{sec:rel}
The merits of applying fusion to document lists retrieved for queries
representing the same information need is now well
understood~\cite{Belkin+al:93a,Belkin+al:95a,Pickens+al:08a,bailey2017retrieval,bc17-adcs,Benham+al:19a}.
One of the retrieval templates we propose uses cluster-based retrieval
to re-rank a list fused from the documents retrieved for a set of
queries representing the same information need.  We also explore the
merits of applying fusion after each of the lists retrieved for the
queries was re-ranked using cluster-based retrieval.  Benham and
Culpepper \cite{bc17-adcs} showed that reciprocal rank fusion
\cite{Cormack+al:09a} consistently yields state-of-the-art performance
when fusing document lists retrieved for multiple queries representing
the same information need.  Our best performing methods substantially
outperform this fusion approach. We note that reciprocal rank fusion
is an unsupervised fusion method. One could potentially further
improve the performance of our methods, some of which rely on fusion
of document lists, by using supervised fusion (e.g.,
\cite{Sheldon+al:11a}). A case in point, one of our two best
performing methods applies fusion to lists which are re-ranked by a
cluster-based retrieval method. Improving the fusion technique with
respect to the reciprocal rank fusion with which we currently instantiate our
templates can potentially increase the effectiveness of this method. We leave this exploration for future work.

%\shane{We should be a little careful here as someone will almost
%certainly point out that this is the best unsupervised method.
%We have never done it, but supervised techniques almost certainly
%perform better such as LambdaMerge which we never got to work
%properly.
%We could add the numbers in the table from the best performing TREC
%run for the Robust collection.
%We cannot do that for CW12.}

Recent work on query performance prediction showed that the relative
prediction quality posted by prediction methods can significantly
vary when varying the effectiveness of a query used to represent an
information need \cite{Zendel+al:19a}.
In a conceptually similar vein, we show that state-of-the-art
cluster-based document retrieval methods can actually be outperformed
by standard bag-of-words document retrieval models if highly
effective queries are used to represent information needs.

To address the efficiency costs of using multiple queries, Benham et
al.~\cite{Benham+al:19a} proposed a highly effective reranking
technique which clusters queries offline.
At query time, the results retrieved for the (single) query input
into the search engine were fused with a centroid of the cluster with
which the query was associated.
The best query cluster to use for reranking is easily identified at
runtime using an auxiliary index of previously seen queries.
Our work focuses on clustering documents at query time, in contrast
to clustering queries offline \cite{Benham+al:19a}.


There was some work on cluster-based fusion of retrieved lists
\cite{Kozorovitzky+Kurland:11b}. Documents in lists retrieved by applying different
retrieval approaches for a {\em single query} were
clustered. Then, a cluster-based retrieval approach was applied using
these clusters to perform the fusion. This fusion approach is easily derived as a specific instantiation of one of the retrieval templates we present for retrieval using {\em multiple queries}. While the resultant performance is effective, we present approaches which are substantially more effective in our setting.



\endinput

Previous work on cluster-based fusion of lists retrieved by different
search systems for a single query used the documents in retrieved
lists for clustering

along with multiple ranking models for a
{\em single query} to induce cluster~\cite{Kozorovitzky+Kurland:11b}.
\shane{Did I get this right?
Sorry the initial text was a little unclear.}
Then, a cluster-based retrieval approach was applied using these
clusters to perform the fusion.
This fusion approach can be derived as a specific instantiation of
one of the retrieval templates we present for retrieval using {\em
multiple queries}.
While the performance of this adaptation is in fact highly effective,
we show that the performance can be further improved in this work.

