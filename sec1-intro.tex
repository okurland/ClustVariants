\section{Introduction}\label{sec-intro}

One of the most fundamental and theoretically sound approaches for
inducing a rich information-need representation in the ad hoc
retrieval task is based on the generative theory of
relevance~\cite{Lavrenko+Croft:03a}.
Specifically, by the generative assumption for relevance, there
exists a relevance language model that generates terms in the query
and in documents relevant to the query, or more precisely, to the
information need it represents.
Relevance models can be estimated using several common techniques
\cite{Lavrenko+Croft:03a,Abdul-Jaleel+al:04a,Lv+Zhai:10a,Seo+Croft:10a}.
The basic approaches can also be extended and improved by combining
multiple information sources; e.g., external corpora
\cite{Diaz+Metzler:06a}, query logs \cite{bmc12-wsdm} and entity
repositories \cite{dda14-sigir}.

None of these prior approaches considers how to directly extend a
relevance model when multiple query variations {\em for the same
information need} are available.
Query variants can easily be gathered through query reformulations
from a user in a single search session, across multiple search
sessions, through query log analysis~{\cite{wnz01-www,sssc11wsdm}},
or through combinations of all of the above.

In this work, we extend the generative assumption of relevance by
assuming that a single language model can generate terms in multiple
queries representing a single information need, and explore the
theoretical and practical implications of this novel extension.
Our extended assumption leads to the formal development of several
new relevance-model estimation methods.

An important aspect of our proposed approaches is {\em data fusion}.
More specifically, our new relevance models can be recast to fusion
at the term level, query-model level (language-model-level), or the
document level.
We formally demonstrate equivalences between several of these methods
despite appearing quite different at first glance.
For example, some of the estimation methods that fuse query models
are equivalent, given some mild assumptions, to a method that
performs fusion at the document level.
These equivalences motivate entirely new relevance-model estimation
approaches that utilize query-model level techniques originally
proposed for fusing document lists, and which have been shown to be
highly effective.

\myparagraph{Contributions}
%
\noindent Our contributions can be summarized as follows: 
(1) We explore a novel task: relevance modeling using multiple query
variations representing the same information need.
(2) We extend the generative assumption for relevance, and use it as
a basis to formally derive relevance-model-estimation methods.
%document level.
(3) We formally demonstrate theoretical connections and equivalences
between several of our methods.
(4) We empirically validate the performance of our proposed
approaches using three different TREC datasets.
We show that models derived from multiple queries are superior to
using a single-query-based model in every case, and demonstrate that a wide
variety of different model combinations exhibit similar performance
characteristics -- providing empirical evidence that our
theoretically derived equivalences also hold in practice.

