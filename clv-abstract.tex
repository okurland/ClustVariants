\begin{abstract}
%The retrieval merits of using multiple queries representing the same
  %information need have long been demonstrated.
  The merits of using
  multiple queries representing the same information need to improve
  retrieval effectiveness have recently been demonstrated in several
  IR studies. In this paper we present the first study of utilizing
  multiple queries in cluster-based document retrieval; that is, using
  information induced from clusters of similar documents to rank
  documents. Specifically, we propose a conceptual framework of
  {\em retrieval templates} that can adapt cluster-based document
  retrieval methods, originally devised for a single query, to
  leverage multiple queries. The adaptations operate at the query,
  document list and feature levels. Retrieval methods are instantiated
  from the templates by selecting, for example, the clustering
  algorithm and the cluster-based retrieval method. Empirical
  evaluation attests to the merits of the retrieval templates with
  respect to very strong baselines: state-of-the-art cluster-based
  retrieval with a single query and highly effective fusion of
  document lists retrieved for multiple queries. In addition, we
  present findings about the impact of the effectiveness of queries used
  to represent an information need on (i) {\em cluster hypothesis}
  test results, (ii) percentage of relevant documents in clusters of
  similar documents, and (iii) effectiveness of state-of-the-art
  cluster-based retrieval methods.
\end{abstract}
